\documentclass[a4paper,twoside,12pt,AutoFakeBold]{ctexart}
\newcommand\specialsectioning{\setcounter{secnumdepth}{-2}}
\specialsectioning
\usepackage[center]{titlesec}
\usepackage{indentfirst}
\setlength{\parindent}{2em}
\usepackage{fancyhdr}
\pagestyle{fancy}%fancy style

\fancyhf{}%清空页眉页脚
\fancyhead[LE,RO]{\thepage}%页码位置:偶数页居左,奇数页居右
\fancyfoot[RO,RE]{\textit{NEFU's seminar of Marxism}}% 设置页脚:在每页的右下脚以斜体显示书名
\usepackage{graphicx}
\setlength{\headheight}{15pt}%解决页眉warnings
\usepackage{amsmath}
\renewcommand{\headrulewidth}{0pt} % 页眉与正文之间的水平线粗细
\renewcommand{\footrulewidth}{0pt}

\usepackage{changepage}%设置引用段落左右侧缩进

\usepackage{tabularx}
\usepackage{hyperref}%设置超链接
\usepackage{float}
\title{读书会记录}
\author{东北林业大学马克思主义研讨会\\
 吴云飞~编}
\date{2023-10}
\usepackage{perpage}
\MakePerPage{footnote}
\begin{document}
\maketitle
\newpage



\tableofcontents%目录

\newpage

\section{序言}

\begin{adjustwidth}{2em}{2em}
\qquad\fangsong 
本书是东北林业大学马克思主义经典著作研究与讨论会的一个内部记录,诸多观点与看法可能会欠缺专业性,因而这是一个非教学性质的内容记录。本书全部内容均来自我们的研讨会中的发言,如有雷同,纯属巧合,本书最终解释权归本研讨会与编辑组所有\footnote{已将本书的\LaTeX{}源代码置于编者的Github之中,供各位成员获取:\url{https://github.com/Imheaven233/Seminar-of-Marxism-of-NEFU}。}。

\end{adjustwidth}




\newpage

\section{第一期:《哲学的贫困》选读(1)}

\subsection{学习提示}\label{sec:1}

《哲学的贫困》是马克思针对蒲鲁东的《贫困的哲学》一书而写的一部论战性著作,以法文写成于1847年上半年,并于同年7月在布鲁塞尔和巴黎出版。

该著作分为两个部分,即第一章和第二章。第一章的讨论针对蒲鲁东为“工资平等”的社会主义所作的经济学论证,揭示这种论证尚未达到李嘉图经济学理论的水准。第二章批判了蒲鲁东经济学理论的哲学基础。

本期讨论会我们选的内容是第二章“\textbf{政治经济学的形而上学}”的第一节“\textbf{方法}”,这部分内容在我们编排的讲义中的3—17页\footnote{这部分内容收录于马恩选集第1卷。}。

在本次讨论会中,您将会了解到(或带着以下问题去阅读):

\textbf{1.}马克思对黑格尔的辩证法的一个简要概述。

\textbf{2.}蒲鲁东的经济学理论所遵循的“辩证法”同黑格尔的辩证法之间的差异。

\textbf{3.}蒲鲁东的经济学理论的哲学基础之实质是什么,犯了什么错误?

\textbf{4.}对蒲鲁东的经济学理论的哲学基础的批判对于我们有何启示?

\subsection{读书会记录}
2023年10月16日18:00—20:00,我们在东北林业大学奥林学院203教室开展了第一期的读书会。本期读书会我们阅读了《哲学的贫困》第二章第一节的部分内容(截止到“\textbf{第五个说明}”之前),并做了充分的讨论。具体讨论内容可简要地概括如下:

\subsubsection{1.前言部分}\label{sec:3}

首先,马克思在前言部分指出:
\begin{adjustwidth}{2em}{2em}
\qquad\fangsong
“如果说有一个英国人把人变成帽子,那末,有一个德国人就把帽子变成观念。这个英国人就是李嘉图,一位银行巨子,杰出的经济学家;这个德国人就是黑格尔,柏林大学的一位专任哲学教授。”
\end{adjustwidth}

这段话中的“帽子”、“观念”指的是什么?在这里我们认为,“帽子”指的是李嘉图的经济学中的诸多“经济范畴”,体现着物与物之间的关系;而“观念”则指的是黑格尔哲学中的“概念”,更确切地说,是黑格尔哲学中的形而上学传统\footnote{这在后面的讨论中会详细的说明。}。

古典经济学家们\footnote{这里指的是英国古典经济学家斯密、李嘉图等人。}通过对资本主义社会的外部经验现象的考察总结出了一些基本的经济规律,但是他没有意识到这些规律本身并非是永恒不变的存在物,以至于他们将人与人之间关系转变为物与物之间的关系,企图通过不变的经济范畴(帽子)解释人类社会的运行本身\footnote{按照马克思的话就是:“把人变成帽子。”}。而持有着形而上学传统的哲学家们(尤指黑格尔)则将这些“帽子”进行了更进一步的“抽象”,将其转化为观念本身\footnote{按照马克思的话就是:“把帽子变成观念。”}。

\subsubsection{2.第一个说明}

第一个说明所讨论的核心内容是黑格尔的辩证法,而黑格尔的辩证法在整体上也是建筑于形而上学这一传统的基础之上的。马克思指出,经济学家们只是向我们解释了生产如何在现存的关系下进行,但是没有向我们解释这些“关系”本身是何以存在的。古典经济学家们面对的是活生生的现实,他们所做的工作是对“活生生的现实”的直接抽象;而蒲鲁东面对的则是古典经济学家们提出的诸多范畴,以及隐藏在这些范畴背后的诸多古典经济学家们的教条与偏见,蒲鲁东在探寻这些“关系”得以产生的原因的过程中,他仅仅从这些范畴本身出发,因而他只能在观念的领域兜圈子。

黑格尔哲学的第一步便是形而上学式的抽象,将一切事物都抽象成为逻辑范畴,并将一切事物的运动也抽象成纯粹形式的运动,这样一来,正如马克思所言:

\begin{adjustwidth}{2em}{2em}
    \qquad\fangsong
    “既然把任何一种事物都归结为逻辑范畴,任何一个运动、任何一种生产行为都归结为方法,那末,由此自然得出一个结论,产品和生产、对象和运动的任何总和都可以归结为应用的形而上学。黑格尔为宗教、法等做过的事情,蒲鲁东先生也想在政治经济学上如法炮制。”
\end{adjustwidth}

由此,黑格尔找到了一种“绝对的方法”来描绘现实世界的运动,即纯粹理性的运动。事实上,这是一种颠倒,但在这里我们就不赘述了\footnote{唯心主义与唯物主义之间的颠倒。}。在这里不难理解,当事物的一切“偶性”都被抽掉之后,剩下的就是纯粹的“概念”、纯粹的“理性”,但需要指出的是,这种纯粹的“概念”与“理性”是脱离个别主体而存在的,是一种绝对的、客观的东西。因此,对于黑格尔的辩证法而言,辨证运动是“绝对精神”的自我运动,是“概念”的自我规定,因而这是一种客观的唯心主义。


\subsubsection{3.第二个说明}

在这部分内容中的末尾,马克思表明了他的历史唯物主义哲学思想:

\begin{adjustwidth}{2em}{2em}
    \qquad\fangsong
    “经济学家蒲鲁东先生非常明白,人们是在一定的生产关系范围内制造呢绒、麻布和丝织品的。但是他不明白,这些一定的社会关系同麻布、亚麻等一样,也是人们生产出来的。社会关系和生产力密切相联。随着新生产力的获得,人们改变自己的生产方式,随着生产方式即保证自己生活的方式的改变,人们也就会改变自己的一切社会关系。手工磨产生的是封建主为首的社会,蒸汽磨产生的是工业资本家为首的社会。”
\end{adjustwidth}

事实上,上面这段话的第一句可以转译为:

\begin{adjustwidth}{2em}{2em}
    \qquad\fangsong
    蒲鲁东明白,人们在一定的“关系”内制造“物”。但是他不明白,这些“关系”同这些“物”一样,也是人们生产出来的。
\end{adjustwidth}
这充分说明了马克思的历史唯物主义哲学与那种纯粹被动的、毫无生机的机械唯物主义哲学之间的差异,因为马克思告诉我们,“关系”可不是什么神秘主义式的东西,“关系”本身就是通过人类的实践活动所建构出来的。

\subsubsection{4.第三个说明}

在“第三个说明”中,马克思揭示了蒲鲁东经济理论的矛盾。马克思是这么说的:

\begin{adjustwidth}{2em}{2em}
    \qquad\fangsong
    “每一个社会中的生产关系都形成一个统一的整体。蒲鲁东先生把种种经济关系看做同等数量的社会阶段,认为这些阶段一个产生一个,一个来自一个,正如反题来自正题一样;认为这些阶段在自己的逻辑顺序中实现着人类的无人身的理性。

这个方法的唯一短处就是:蒲鲁东先生在考察其中任何一个阶段时,都不能不靠其它一些社会关系来说明,可是当时这些社会关系尚未被他用辩证运动产生出来。当蒲鲁东先生后来借助纯粹理性使其它阶段产生出来时,却又把它们当成初生的婴儿,忘记它们和第一个阶段是同样年老了。

因此,要构成被他看做一切经济发展基础的价值,就非有分工、竞争等等不可。然而当时这些关系在一定的系列中、在蒲鲁东先生的理性中以及逻辑顺序中根本还不存在。”
\end{adjustwidth}
在这里可以看到,蒲鲁东的“辩证法”很有意思,当他运用“辩证法”推出某些新的概念时,他总是需要依靠一些尚未被他的“辩证法”所生成出来的东西去说明。例如,蒲鲁东通过A和B推出了C,再通过C和I推出了D,但是I本身并没有通过他的辩证法被创造出来,而当他通过E和F推出I的时候,他也忘记了,I本身在C到D的“辩证运动”中就已经作为前提条件生成了。因此,马克思在这里隐约地想表明,蒲鲁东的“辩证法”似乎是一种非辩证的主观臆想。

\subsubsection{5.第四个说明}

通过对前面部分的阅读与讨论,“第四个说明”这部分内容便显得容易理解了。马克思在这一部分中说明了蒲鲁东是如何将他的“辩证法”应用到政治经济学领域的。马克思是这么叙述的:

\begin{adjustwidth}{2em}{2em}
    \qquad\fangsong
“蒲鲁东先生认为,任何经济范畴都有好坏两个方面。他看范畴就象小资产者看历史伟人一样:拿破仑是一个大人物;他行了许多善,但是也作了许多恶。

蒲鲁东先生认为,好的方面和坏的方面,益处和害处加在一起就构成每个经济范畴所固有的矛盾。

应当作的是:保存好的方面,消除坏的方面。”
\end{adjustwidth}

可见,当蒲鲁东的“辩证法”应用到政治经济学的领域时便成了一种机械式的“保存好的方面,消除坏的方面”。

因此,马克思会指出:

\begin{adjustwidth}{2em}{2em}
\qquad\fangsong
    “黑格尔没有需要提出任务。他只有辩证法。蒲鲁东先生从黑格尔的辩证法那里只学到了术语。而蒲鲁东先生自己的辨证运动只不过是机械地划分出好、坏两面而已。
    
    ……
    
    两个矛盾方面的共存、斗争以及融合成一个新范畴,就是辩证运动的实质。谁要给自己提出消除坏的方面的任务,就是立即使辩证运动终结。我们看到的已经不是由于矛盾本性而自我安置和自相对置的范畴,而是在范畴的两个方面中间激动、挣扎和冲撞的蒲鲁东先生。”

\end{adjustwidth}
    
\subsubsection{简短的总结}

通过本期的研讨会,我们可以对在\textbf{\nameref{sec:1}}中提出的四个问题中的前两个进行简要地回答。

对于\textbf{问题1“马克思对黑格尔的辩证法的一个简要概述”}而言,我们可以认为黑格尔的辩证法表明的是概念的自我规定(或自我运动),且这种辩证运动的前提基础是形而上学式的抽象。

对于\textbf{问题2“蒲鲁东的经济学理论所遵循的“辩证法”同黑格尔的辩证法之间的差异”}而言,我们可以认为黑格尔的辩证法所揭露的是脱离于单个主体之外的纯粹理性的自我运动,是一种客观唯心主义;而蒲鲁东的“辩证法”仅仅是借用了黑格尔哲学的诸多词句,他并没有领会黑格尔辩证法的实质,他将黑格尔的辩证法下降到了简单的二元式的机械运动,更具体地说,可以认为蒲鲁东的“辩证法”是一种主观唯心主义。


\end{document}
